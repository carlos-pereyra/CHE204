%\documentclass{scrartcl}
\documentclass{article}

\usepackage{geometry} % document size
 \geometry{
 textheight = 700 pt,
 footskip = 80 pt
 }

%% strictly for grid/tikz figures
\usepackage{script1}  %% grid diagrams formatting
\usepackage{script2}  %% code formatting

%% strictly for fancy headers
\usepackage{fancyhdr} % fancy headers
\pagestyle{fancy}
\fancyhf{}
\rhead{Homework 5}
\lhead{}
\rfoot{Page \thepage}

\newcommand{\tmpx}{}
\newcommand\tmp[1]{\renewcommand{\tmpx}{#1}}
\fancypagestyle{sec}{\lhead{\tmpx}}

%% strictly for basic math operations
\usepackage{amsmath,amsthm,amssymb}
\usepackage{mathtools}
\usepackage{graphicx}
\usepackage{caption}
\usepackage{subcaption}
\usepackage{braket}
\usepackage{epstopdf}
\usepackage{float} % hold image in section

%% strictly for tables
\usepackage{tabularx} % tables

%% strictly for including code snippets
\usepackage{listings}	% display code
\usepackage{xcolor}
\definecolor{codegreen}{rgb}{0,0.6,0}
\definecolor{codegray}{rgb}{0.5,0.5,0.5}
\definecolor{codepurple}{rgb}{0.58,0,0.82}
\definecolor{backcolour}{rgb}{0.95,0.95,0.92}

\lstdefinestyle{mystyle}{
    backgroundcolor=\color{backcolour},   
    commentstyle=\color{codegreen},
    keywordstyle=\color{magenta},
    numberstyle=\tiny\color{codegray},
    stringstyle=\color{codepurple},
    basicstyle=\ttfamily\footnotesize,
    breakatwhitespace=false,         
    breaklines=true,                 
    captionpos=b,                    
    keepspaces=true,                 
    numbers=left,                    
    numbersep=5pt,                  
    showspaces=false,                
    showstringspaces=false,
    showtabs=false,                  
    tabsize=2
}
%\lstset{style=mystyle}

\usepackage{mathptmx}
\usepackage[scaled=0.92]{helvet}
\usepackage{courier}

%% For piecewise defined functions
\DeclarePairedDelimiter\Floor\lfloor\rfloor
\DeclarePairedDelimiter\Ceil\lceil\rceil

%% BOXED EQUATIONS FORMAT
\usepackage{tcolorbox}
\newtcolorbox{mybox}[1][]{sharp corners, boxsep=3pt, #1}

%% MATRIX NOTATION
\def\doubleunderline#1{\underline{\underline{#1}}}

\begin{document}
 
 \title{CHE 204: Homework 5}
 %\subtitle{}
 \author{Carlos Pereyra}
 \date{December, 2020}
 \maketitle
 
 % \begin{lstlisting}[language=R]
 % \end{lstlisting}
 \section{System of differential equations}
\textit{Consider the modified Lotka Volterra system that describes a prey-predator model. The equation for the population of preys has been changed to consider limited growth due to finite resources (the term $u^2$):}\\
\begin{align}
\frac{du}{dt} &= (\alpha u - \alpha' u^{2}) - \beta u v \\
\frac{dv}{dt} &= -\gamma v + \delta u v
\end{align}
\\
\textit{with the parameters $\alpha = 4$, $\alpha' = 2$, $\beta = 1$,$\gamma = 2$, and $\delta = 2$.}

\subsection{Stability}
\textit{Discuss the stability of the stationary solutions, by linearizing the system of differential equations. (analytical)}
\\ \\
So for the sake of easier readability, I have done a change of variables so $u$ and $v$ are not to be confused. Personally I know they can be mixed up, for I might be partially dyslexic. So now let $u=y_1$ and $v=y_2$.

\begin{align}
\label{eqn:base1} F_1 = \frac{dy_1}{dt} &= (\alpha y_1 - \alpha' y_1^2) - \beta y_1 y_2\\
 F_2 = \frac{dy_2}{dt} &= -\gamma y_2 + \delta y_1 y_2
\end{align}

Through a moderate amount of tedious algebra we find the critical points where we let $F_1=0$ and $F_2=0$ and solve for the roots. Finally we conclude with three critical points where the trivial solution (0,0) is of course among them. Here are the following critical points: 
\\
\begin{mybox}[standard jigsaw, opacityback=0, title=critical points, colframe=black!30!black]
\begin{align}
(\tilde{y}_1^*, \tilde{y}_2^*) \rightarrow &(0,0) \\[6pt]
(\tilde{y}_1^*, \tilde{y}_2^*) \rightarrow (\frac{\alpha}{\alpha'}, 0) \rightarrow &(2,0) \\[6pt]
(\tilde{y}_1^*, \tilde{y}_2^*) \rightarrow (\frac{\gamma}{\delta}, \frac{\alpha - \alpha'\left(\frac{\gamma}{\delta}\right)}{\beta}) \rightarrow &(1,2)
\end{align}
\end{mybox}

Obviously equations \eqref{eqn:base1} and \eqref{eqn:base1} are non-linear, so the system differential equations here can be represented like so, $\underline{\textbf{Y'}}=\doubleunderline{\textbf{A}}\ \underline{\textbf{Y}}+ \underline{\textbf{h}}(\underline{\textbf{Y}})$. So the determinant of \doubleunderline{\textbf{A}}

\begin{table}[H]
\centering
\begin{tabularx}{\columnwidth}{XX}
    \centering
    \resizebox{!}{.15\paperheight}{\input{problem1/criticalpoint1/image.tex}}
    %\input{problem1/criticalpoint1/fig.ps}
    \captionof{figure}{$(\tilde{y}_1^*, \tilde{y}_2^*) = (2,0)$}\label{fig:StabilityPlot1}
    &
    \centering
    \resizebox{!}{.15\paperheight}{\input{problem1/criticalpoint2/image.tex}}
    \captionof{figure}{$(\tilde{y}_1^*, \tilde{y}_2^*) = (1,2)$}\label{fig:StabilityPlot2}
\end{tabularx}
\end{table}

\begin{figure}[H]
\centering
\begin{tikzpicture}

    \gridThreeD{0}{10}{black!50}{8}{8}{4}

    \gridThreeD{0}{4.25}{black!50}{8}{8}{2}

    \gridThreeD{0}{0}{black!50}{8}{8}{1}
    
    %draws lower graph lines and those in z-direction:
    \begin{scope}
        \myGlobalTransformation{0}{0};
        %\graphLinesHorizontal;

        %draws all graph lines in z-direction (reset transformation first!):
        \foreach \x in {2,4,6} {
            \foreach \y in {2,4,6} {
                \node (thisNode) at (\x,\y) {};
                {
                    \pgftransformreset
                    \draw[white,myBG]  (thisNode) -- ++(0,4.25);
                    \draw[black,very thick] (thisNode) -- ++(0,4.25);
                }
            }
        }
    \end{scope}
    
    % draws all graph nodes:
    \graphThreeDnodes{0}{0};
    \graphThreeDnodes{0}{4.25};
    
\end{tikzpicture}
\end{figure}






\end{document}